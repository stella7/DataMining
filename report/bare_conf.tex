
%% bare_conf.tex
%% V1.4b
%% 2015/08/26
%% by Michael Shell
%% See:
%% http://www.michaelshell.org/
%% for current contact information.
%%
%% This is a skeleton file demonstrating the use of IEEEtran.cls
%% (requires IEEEtran.cls version 1.8b or later) with an IEEE
%% conference paper.
%%
%% Support sites:
%% http://www.michaelshell.org/tex/ieeetran/
%% http://www.ctan.org/pkg/ieeetran
%% and
%% http://www.ieee.org/

%%*************************************************************************
%% Legal Notice:
%% This code is offered as-is without any warranty either expressed or
%% implied; without even the implied warranty of MERCHANTABILITY or
%% FITNESS FOR A PARTICULAR PURPOSE! 
%% User assumes all risk.
%% In no event shall the IEEE or any contributor to this code be liable for
%% any damages or losses, including, but not limited to, incidental,
%% consequential, or any other damages, resulting from the use or misuse
%% of any information contained here.
%%
%% All comments are the opinions of their respective authors and are not
%% necessarily endorsed by the IEEE.
%%
%% This work is distributed under the LaTeX Project Public License (LPPL)
%% ( http://www.latex-project.org/ ) version 1.3, and may be freely used,
%% distributed and modified. A copy of the LPPL, version 1.3, is included
%% in the base LaTeX documentation of all distributions of LaTeX released
%% 2003/12/01 or later.
%% Retain all contribution notices and credits.
%% ** Modified files should be clearly indicated as such, including  **
%% ** renaming them and changing author support contact information. **
%%*************************************************************************


% *** Authors should verify (and, if needed, correct) their LaTeX system  ***
% *** with the testflow diagnostic prior to trusting their LaTeX platform ***
% *** with production work. The IEEE's font choices and paper sizes can   ***
% *** trigger bugs that do not appear when using other class files.       ***                          ***
% The testflow support page is at:
% http://www.michaelshell.org/tex/testflow/



\documentclass[conference]{IEEEtran}
% Some Computer Society conferences also require the compsoc mode option,
% but others use the standard conference format.
%
% If IEEEtran.cls has not been installed into the LaTeX system files,
% manually specify the path to it like:
% \documentclass[conference]{../sty/IEEEtran}





% Some very useful LaTeX packages include:
% (uncomment the ones you want to load)


% *** MISC UTILITY PACKAGES ***
%
%\usepackage{ifpdf}
% Heiko Oberdiek's ifpdf.sty is very useful if you need conditional
% compilation based on whether the output is pdf or dvi.
% usage:
% \ifpdf
%   % pdf code
% \else
%   % dvi code
% \fi
% The latest version of ifpdf.sty can be obtained from:
% http://www.ctan.org/pkg/ifpdf
% Also, note that IEEEtran.cls V1.7 and later provides a builtin
% \ifCLASSINFOpdf conditional that works the same way.
% When switching from latex to pdflatex and vice-versa, the compiler may
% have to be run twice to clear warning/error messages.






% *** CITATION PACKAGES ***
%
%\usepackage{cite}
% cite.sty was written by Donald Arseneau
% V1.6 and later of IEEEtran pre-defines the format of the cite.sty package
% \cite{} output to follow that of the IEEE. Loading the cite package will
% result in citation numbers being automatically sorted and properly
% "compressed/ranged". e.g., [1], [9], [2], [7], [5], [6] without using
% cite.sty will become [1], [2], [5]--[7], [9] using cite.sty. cite.sty's
% \cite will automatically add leading space, if needed. Use cite.sty's
% noadjust option (cite.sty V3.8 and later) if you want to turn this off
% such as if a citation ever needs to be enclosed in parenthesis.
% cite.sty is already installed on most LaTeX systems. Be sure and use
% version 5.0 (2009-03-20) and later if using hyperref.sty.
% The latest version can be obtained at:
% http://www.ctan.org/pkg/cite
% The documentation is contained in the cite.sty file itself.






% *** GRAPHICS RELATED PACKAGES ***
%
\ifCLASSINFOpdf
  % \usepackage[pdftex]{graphicx}
  % declare the path(s) where your graphic files are
  % \graphicspath{{../pdf/}{../jpeg/}}
  % and their extensions so you won't have to specify these with
  % every instance of \includegraphics
  % \DeclareGraphicsExtensions{.pdf,.jpeg,.png}
\else
  % or other class option (dvipsone, dvipdf, if not using dvips). graphicx
  % will default to the driver specified in the system graphics.cfg if no
  % driver is specified.
  % \usepackage[dvips]{graphicx}
  % declare the path(s) where your graphic files are
  % \graphicspath{{../eps/}}
  % and their extensions so you won't have to specify these with
  % every instance of \includegraphics
  % \DeclareGraphicsExtensions{.eps}
\fi
% graphicx was written by David Carlisle and Sebastian Rahtz. It is
% required if you want graphics, photos, etc. graphicx.sty is already
% installed on most LaTeX systems. The latest version and documentation
% can be obtained at: 
% http://www.ctan.org/pkg/graphicx
% Another good source of documentation is "Using Imported Graphics in
% LaTeX2e" by Keith Reckdahl which can be found at:
% http://www.ctan.org/pkg/epslatex
%
% latex, and pdflatex in dvi mode, support graphics in encapsulated
% postscript (.eps) format. pdflatex in pdf mode supports graphics
% in .pdf, .jpeg, .png and .mps (metapost) formats. Users should ensure
% that all non-photo figures use a vector format (.eps, .pdf, .mps) and
% not a bitmapped formats (.jpeg, .png). The IEEE frowns on bitmapped formats
% which can result in "jaggedy"/blurry rendering of lines and letters as
% well as large increases in file sizes.
%
% You can find documentation about the pdfTeX application at:
% http://www.tug.org/applications/pdftex





% *** MATH PACKAGES ***
%
%\usepackage{amsmath}
% A popular package from the American Mathematical Society that provides
% many useful and powerful commands for dealing with mathematics.
%
% Note that the amsmath package sets \interdisplaylinepenalty to 10000
% thus preventing page breaks from occurring within multiline equations. Use:
%\interdisplaylinepenalty=2500
% after loading amsmath to restore such page breaks as IEEEtran.cls normally
% does. amsmath.sty is already installed on most LaTeX systems. The latest
% version and documentation can be obtained at:
% http://www.ctan.org/pkg/amsmath





% *** SPECIALIZED LIST PACKAGES ***
%
%\usepackage{algorithmic}
% algorithmic.sty was written by Peter Williams and Rogerio Brito.
% This package provides an algorithmic environment fo describing algorithms.
% You can use the algorithmic environment in-text or within a figure
% environment to provide for a floating algorithm. Do NOT use the algorithm
% floating environment provided by algorithm.sty (by the same authors) or
% algorithm2e.sty (by Christophe Fiorio) as the IEEE does not use dedicated
% algorithm float types and packages that provide these will not provide
% correct IEEE style captions. The latest version and documentation of
% algorithmic.sty can be obtained at:
% http://www.ctan.org/pkg/algorithms
% Also of interest may be the (relatively newer and more customizable)
% algorithmicx.sty package by Szasz Janos:
% http://www.ctan.org/pkg/algorithmicx




% *** ALIGNMENT PACKAGES ***
%
%\usepackage{array}
% Frank Mittelbach's and David Carlisle's array.sty patches and improves
% the standard LaTeX2e array and tabular environments to provide better
% appearance and additional user controls. As the default LaTeX2e table
% generation code is lacking to the point of almost being broken with
% respect to the quality of the end results, all users are strongly
% advised to use an enhanced (at the very least that provided by array.sty)
% set of table tools. array.sty is already installed on most systems. The
% latest version and documentation can be obtained at:
% http://www.ctan.org/pkg/array


% IEEEtran contains the IEEEeqnarray family of commands that can be used to
% generate multiline equations as well as matrices, tables, etc., of high
% quality.




% *** SUBFIGURE PACKAGES ***
%\ifCLASSOPTIONcompsoc
%  \usepackage[caption=false,font=normalsize,labelfont=sf,textfont=sf]{subfig}
%\else
%  \usepackage[caption=false,font=footnotesize]{subfig}
%\fi
% subfig.sty, written by Steven Douglas Cochran, is the modern replacement
% for subfigure.sty, the latter of which is no longer maintained and is
% incompatible with some LaTeX packages including fixltx2e. However,
% subfig.sty requires and automatically loads Axel Sommerfeldt's caption.sty
% which will override IEEEtran.cls' handling of captions and this will result
% in non-IEEE style figure/table captions. To prevent this problem, be sure
% and invoke subfig.sty's "caption=false" package option (available since
% subfig.sty version 1.3, 2005/06/28) as this is will preserve IEEEtran.cls
% handling of captions.
% Note that the Computer Society format requires a larger sans serif font
% than the serif footnote size font used in traditional IEEE formatting
% and thus the need to invoke different subfig.sty package options depending
% on whether compsoc mode has been enabled.
%
% The latest version and documentation of subfig.sty can be obtained at:
% http://www.ctan.org/pkg/subfig




% *** FLOAT PACKAGES ***
%
%\usepackage{fixltx2e}
% fixltx2e, the successor to the earlier fix2col.sty, was written by
% Frank Mittelbach and David Carlisle. This package corrects a few problems
% in the LaTeX2e kernel, the most notable of which is that in current
% LaTeX2e releases, the ordering of single and double column floats is not
% guaranteed to be preserved. Thus, an unpatched LaTeX2e can allow a
% single column figure to be placed prior to an earlier double column
% figure.
% Be aware that LaTeX2e kernels dated 2015 and later have fixltx2e.sty's
% corrections already built into the system in which case a warning will
% be issued if an attempt is made to load fixltx2e.sty as it is no longer
% needed.
% The latest version and documentation can be found at:
% http://www.ctan.org/pkg/fixltx2e


%\usepackage{stfloats}
% stfloats.sty was written by Sigitas Tolusis. This package gives LaTeX2e
% the ability to do double column floats at the bottom of the page as well
% as the top. (e.g., "\begin{figure*}[!b]" is not normally possible in
% LaTeX2e). It also provides a command:
%\fnbelowfloat
% to enable the placement of footnotes below bottom floats (the standard
% LaTeX2e kernel puts them above bottom floats). This is an invasive package
% which rewrites many portions of the LaTeX2e float routines. It may not work
% with other packages that modify the LaTeX2e float routines. The latest
% version and documentation can be obtained at:
% http://www.ctan.org/pkg/stfloats
% Do not use the stfloats baselinefloat ability as the IEEE does not allow
% \baselineskip to stretch. Authors submitting work to the IEEE should note
% that the IEEE rarely uses double column equations and that authors should try
% to avoid such use. Do not be tempted to use the cuted.sty or midfloat.sty
% packages (also by Sigitas Tolusis) as the IEEE does not format its papers in
% such ways.
% Do not attempt to use stfloats with fixltx2e as they are incompatible.
% Instead, use Morten Hogholm'a dblfloatfix which combines the features
% of both fixltx2e and stfloats:
%
% \usepackage{dblfloatfix}
% The latest version can be found at:
% http://www.ctan.org/pkg/dblfloatfix




% *** PDF, URL AND HYPERLINK PACKAGES ***
%
%\usepackage{url}
% url.sty was written by Donald Arseneau. It provides better support for
% handling and breaking URLs. url.sty is already installed on most LaTeX
% systems. The latest version and documentation can be obtained at:
% http://www.ctan.org/pkg/url
% Basically, \url{my_url_here}.




% *** Do not adjust lengths that control margins, column widths, etc. ***
% *** Do not use packages that alter fonts (such as pslatex).         ***
% There should be no need to do such things with IEEEtran.cls V1.6 and later.
% (Unless specifically asked to do so by the journal or conference you plan
% to submit to, of course. )


% correct bad hyphenation here
\hyphenation{op-tical net-works semi-conduc-tor}

\usepackage{graphicx}
\begin{document}
%
% paper title
% Titles are generally capitalized except for words such as a, an, and, as,
% at, but, by, for, in, nor, of, on, or, the, to and up, which are usually
% not capitalized unless they are the first or last word of the title.
% Linebreaks \\ can be used within to get better formatting as desired.
% Do not put math or special symbols in the title.
\title{Knowledge Discovery in the Movie Database}


% author names and affiliations
% use a multiple column layout for up to three different
% affiliations
\author{\IEEEauthorblockN{Yuzhou Wang}
\IEEEauthorblockA{School of Electrical and Computer Engineering\\
University of Waterloo\\
Waterloo, ON, Cananda \\
Email: ...@uwaterloo.ca}
\and
\IEEEauthorblockN{Sainan He}
\IEEEauthorblockA{School of Electrical and Computer Engineering\\
	University of Waterloo\\
	Waterloo, ON, Cananda \\
	Email: s66he@uwaterloo.ca}
}

% conference papers do not typically use \thanks and this command
% is locked out in conference mode. If really needed, such as for
% the acknowledgment of grants, issue a \IEEEoverridecommandlockouts
% after \documentclass

% for over three affiliations, or if they all won't fit within the width
% of the page, use this alternative format:
% 
%\author{\IEEEauthorblockN{Michael Shell\IEEEauthorrefmark{1},
%Homer Simpson\IEEEauthorrefmark{2},
%James Kirk\IEEEauthorrefmark{3}, 
%Montgomery Scott\IEEEauthorrefmark{3} and
%Eldon Tyrell\IEEEauthorrefmark{4}}
%\IEEEauthorblockA{\IEEEauthorrefmark{1}School of Electrical and Computer Engineering\\
%Georgia Institute of Technology,
%Atlanta, Georgia 30332--0250\\ Email: see http://www.michaelshell.org/contact.html}
%\IEEEauthorblockA{\IEEEauthorrefmark{2}Twentieth Century Fox, Springfield, USA\\
%Email: homer@thesimpsons.com}
%\IEEEauthorblockA{\IEEEauthorrefmark{3}Starfleet Academy, San Francisco, California 96678-2391\\
%Telephone: (800) 555--1212, Fax: (888) 555--1212}
%\IEEEauthorblockA{\IEEEauthorrefmark{4}Tyrell Inc., 123 Replicant Street, Los Angeles, California 90210--4321}}




% use for special paper notices
%\IEEEspecialpapernotice{(Invited Paper)}




% make the title area
\maketitle

% As a general rule, do not put math, special symbols or citations
% in the abstract
\begin{abstract}
The abundance of movie data in terms of review, rating or even detail information in the internet has encouraged many researches to formulate techniques to analyze the pattern in movie data invovlving in discovering factors which will influence the success of movies and  developping recommendation systems of movie according to user reviews. In this paper, we apply different data mining techniques including association rules, clutering, classification and 
prediction on two Internet movie datasets.

\end{abstract}

% no keywords




% For peer review papers, you can put extra information on the cover
% page as needed:
% \ifCLASSOPTIONpeerreview
% \begin{center} \bfseries EDICS Category: 3-BBND \end{center}
% \fi
%
% For peerreview papers, this IEEEtran command inserts a page break and
% creates the second title. It will be ignored for other modes.
\IEEEpeerreviewmaketitle



\section{Introduction}
% no \IEEEPARstart
Nowadays, we live in a world with vast amounts of data collected everyday. Analyzing such data is an urgent need, thus, data mining has become a popular topic and a fast-growing field. Data mining techniques are already widely deployed in many essential area such as business, society, science, medicine, engineering and almost everywhere\cite{jiawei}.

Data mining involves different knowledge discovery such as classification, clustering, association analysis\cite{Usama}. Association rule learning is a method for discovering interesting relations between variables in large databases. Clustering is an unsupervised data mining technique for discovering interesting patterns from a given database. Classification is a supervised approach classifying records in a data set into predefined classes or even defining classes on the go.

It is very likely that a user will give a movie similar rating with another user who has the same taste. Such an approach that making predictions based on the interests of a user by collecting preference or tastes information from other users is Collaborative Filtering(CF) which is the most popular method in recommendation systems.
 
The objective of this paper is firstly, to provide a suitable approach along with necessary factors that are to be considered for association rules, clustering and classification using Internet Movie Database (IMDB) data. Performing different classification, a comparison is based on the evaluation the results. Lastly, apply collaborative filtering method to predict users' rating to movies using MovieLens dataset.

The organisation of the paper is as follows: Section 2 provides the literature review about the problem domain. Section 3 and 4 introduces the two dataset that we used in this paper and the data processing approaches. Section 5 gives an overview of the techniques we use to perform our analysis. Section 6 describes the actual analysis performed, and then presents the results and a discussion thereof. Section 7 gives the conclusions reached and a note about possible further work.
% You must have at least 2 lines in the paragraph with the drop letter
% (should never be an issue)


\section{Literature Review}
\subsection{Data mining for the internet of things: literature review and challenges\cite{fengchen}}
This is a review article surveying data mining through 3 perspectives. From the knowledge view, it illustrated classification, clustering, association analysis, time series analysis, outlier analysis and related realization such as SVM, Bayesian networks, SVD, etc from technique view. Then it introduced application of these theory and techniques in different fields. It also discussed challenges for data mining, like extraction useful data from large quantity of data with low quality. Finally, based on the former analysis on data mining algorithms and application, the author proposed a system architecture for big data mining system.
\subsection{A Case Study in a Recommender System Based on Purchase Data\cite{Bruno}}
This paper present three kinds of collaborative filtering algorithms- memory based approach, matrix factorization and bigram matrix method- on a real-world dataset for recommending items to customer according customers’ purchase histories. The author established models for the three methods and applied them on different settings, comparing their results. They also proposed the multidimensional model for contextual analysis, but they didn’t talk about it in detail due to the space limit. The research mainly drew the conclusion that: (1) the algorithm based on bigram association rules obtained the best performances; (2) the performance of these algorithms has slight difference compared to that of introducing contextual analysis.
\subsection{Algorithms and Methods in Recommender Systems\cite{Daniar}}
In this report, the researcher have described traditional and modern recommender approaches with giving concrete examples and presenting their problems. The traditional ones which work with profiles of users, Content-based filtering measuring similarities and Collaborative  filtering building neighbourhood, are usually combined to avoid some limitations and problems for better results. Some modern methods are an extension of collaborative filtering, such as Context-aware, Semantic-based and Cross-domain based approaches, which outperform original one. Nevertheless, obtaining context information and creating new text mining techniques are some of the problems remaining to be solved. Others like Peer-to-Peer and Cross-lingual approaches are briefly introduced by the researcher.

\section{Data Collection}

\subsection{IMDB}
The Internet Movie Database (IMDb) is a comprehensive online database having information about movies, actors, television shows, production, etc. The IMDb web site\cite{imdb} provides more than 50 text files in ad-hoc format (called lists) containing different characteristics about movies (e.g. actors.list or running-times.list). Given the large scale of the data and the degree of interactions between the people, IMDb is a fertile source of data mining problems.
\subsection{MovieLens}
MovieLens is a movie recommender project, developed by the Department of Computer Science and Engineering at the University of Minnesota. MovieLens is a typical collaborative filtering system that collects movie preferences from users and then groups users with similar tastes. Based on the movie ratings expressed by all the users in a group it attempts to predict for each individual their opinion on movies they have not yet seen. Two data sets are available at the MovieLens web site\cite{movielens}. In this paper, we will use the MovieLens 100K dataset which consists of 100,000 ratings for 1682 movies by 943 users.
\subsection{Subsection Heading Here}
Subsection text here.


\subsubsection{Subsubsection Heading Here}
Subsubsection text here.

\section{Data Preparation}

\section{Methodologys}
\subsection{Association Rules}
\subsection{Clustering}
\subsection{Classification}
\subsection{Collaborative Filtering}

\textit{User-based collaborative filtering}, also know as \textit{k-NN collaborative},was the first of the automated CF methods. It find other users whose past rating behavior is similar to that of current user and
use their ratings on that item to predict what the current user will like. 

\textit{Item-based Collaborative Filtering} is similar to User-based CF, it uses similarity between the rating patterns of items. If two items tend to have the same users like and dislike them,then they are similar and users are expected to have similar preferences for similar items.
\subsubsection{Computing Predictions}
To compute predictions or recommendations for a user u, user-user CF firstly needs to determine the number N of neighbors will be used to generate the result. Then computing the weighted average of the chosen neighboring users' rating i by using similarity as weights. The formula is given as below:

\begin{equation}
p_{u,i} ={\bar r_{u}}  + \frac
{\sum\nolimits_{u' \in N} s(u,u')(r_{u',i} - {\bar r_{u'}})} 
{\sum\nolimits_{u' \in N} |s(u,u')|}
\end{equation}

In order to eliminate the differences in users's use of the rating scale, subtracting the user's mean rating ${\bar r_{u'}}$ to compensate is necessary.   The parameter $p_{u,i}$ is predicated rating on item i for user $u$.  ${\bar r_{u'}}$ is average rating on all items rated by user $u$.The parameter $r_{u',i}$ indicates the rating of user $u'$ on item i.$s(u,u')$ is similarity between user u and $u'$. N is the number of neighbors chosen for user $u$.  

\subsubsection{Measure of Similarity}
An critical parameter used to calculate predications is similarity function. One of the most common and typically measurements is the cosine similarity. 

In Cosine Similarity model, users are represented as $|\textit{I}|$-dimensional vectors of rating on $|\textit{I}|$ items. Similarity is measured by the cosine distance between two rating vectors. The formula is given below indicating how to calculate the Cosine Similarity between user $u$ and $v$. 
\begin{equation}
s(u,v) = \frac{\sum\nolimits_{i} r_{u,i}r_{v,i}}{\sqrt{\sum\nolimits_{i} {r^{2}_{u,i}}} \sqrt{\sum\nolimits_{i} {r^{2}_{v,i}}}} 
\end{equation}

$r_u$ is rating vector of user $u$.

\subsubsection{Evaluation Metrics}
Our goal is to predict the rating a user would give to a restaurant. We predict the rating that user has not rated in the training dataset, but the true rating is stored in the test dataset. We use the root-mean-square error and mean-absolute error for evaluation.
\begin{equation}
\ RMSE = \sqrt{\frac{\sum{\left(r'_{u,i} - r_{u,i}\right)}^2}{N}}
\end{equation}
\begin{equation}
\ MAE = \sqrt{\frac{\sum{\left|r'_{u,i} - r_{u,i}\right|}}{N}}
\end{equation}

Here $r'_{u,i}$ is the predicted rating from user $u$ on item $i$ and $r_{u,i}$ is the true rating; $N$ is the size of test dataset. 

Another evaluation method is precision-recall: precision tells us how good the predictions are. In other words, how many were a hit.; recall tells us how many of the hits were accounted for, or the coverage of the desirable outcome.
\begin{equation}
\ precision = \frac{\left|\{relevant documents\}\bigcap \{retrieved documents\}\right|}{\left|\{retrieved documents\}\right|}
\end{equation}
\begin{equation}
\ recall = \frac{\left|\{relevant documents\}\bigcap \{retrieved documents\}\right|}{\left|\{retrieved documents\}\right|}
\end{equation}


\section{Experimental Evaluation }
\subsection{title}
\subsection{title}

\subsection{Rating Prediction}
Data for this part is from the MovieLens dataset which is rich source for recommendation. We firstly visualize the data by ploting several histograms. Figure 1 is the rating distribution of the raw data. The rating range is from 1 star to 5 stars. Then we use a z-score normalization to normalize the data for further analysis. Figure 2 shows the distribution of user's rating count. We can see that many people just reviews quite a few movies. Figure 3 is the distribution of each movie's average rating. It indicates that most movies receive 3 to 4 stars.
\begin{figure}
	\centering
	\includegraphics[width=2.5in]{rating.png}
	\caption{Rating Distribution}
	\label{fig:side:a}
\end{figure}
\begin{figure}
	\centering
	\includegraphics[width=2.5in]{user_rating.png}
	\caption{User Rating Count Distribution}
	\label{fig:side:a}
\end{figure}

\begin{figure}
	\centering
	\includegraphics[width=2.5in]{movie_rating.png}
	\caption{Movie Average Rating Distribution}
	\label{fig:side:a}
\end{figure}

We apply both user-based CF and item-based CF for rating predicting. Also, we use a random prediction for the baseline.
\begin{figure}
	\centering
	\includegraphics[width=2.8in]{RMSE.png}
	\caption{RMSE}
	\label{fig:side:a}
\end{figure}
\begin{figure}
	\centering
	\includegraphics[width=2.8in]{ROC.png}
	\caption{ROC Curves}
	\label{fig:side:a}
\end{figure}

\begin{figure}
	\centering
	\includegraphics[width=2.8in]{precision_recall.png}
	\caption{Precision and Recall Curves}
	\label{fig:side:a}
\end{figure}
% An example of a floating figure using the graphicx package.
% Note that \label must occur AFTER (or within) \caption.
% For figures, \caption should occur after the \includegraphics.
% Note that IEEEtran v1.7 and later has special internal code that
% is designed to preserve the operation of \label within \caption
% even when the captionsoff option is in effect. However, because
% of issues like this, it may be the safest practice to put all your
% \label just after \caption rather than within \caption{}.
%
% Reminder: the "draftcls" or "draftclsnofoot", not "draft", class
% option should be used if it is desired that the figures are to be
% displayed while in draft mode.
%
%\begin{figure}[!t]
%\centering
%\includegraphics[width=2.5in]{myfigure}
% where an .eps filename suffix will be assumed under latex, 
% and a .pdf suffix will be assumed for pdflatex; or what has been declared
% via \DeclareGraphicsExtensions.
%\caption{Simulation results for the network.}
%\label{fig_sim}
%\end{figure}

% Note that the IEEE typically puts floats only at the top, even when this
% results in a large percentage of a column being occupied by floats.


% An example of a double column floating figure using two subfigures.
% (The subfig.sty package must be loaded for this to work.)
% The subfigure \label commands are set within each subfloat command,
% and the \label for the overall figure must come after \caption.
% \hfil is used as a separator to get equal spacing.
% Watch out that the combined width of all the subfigures on a 
% line do not exceed the text width or a line break will occur.
%
%\begin{figure*}[!t]
%\centering
%\subfloat[Case I]{\includegraphics[width=2.5in]{box}%
%\label{fig_first_case}}
%\hfil
%\subfloat[Case II]{\includegraphics[width=2.5in]{box}%
%\label{fig_second_case}}
%\caption{Simulation results for the network.}
%\label{fig_sim}
%\end{figure*}
%
% Note that often IEEE papers with subfigures do not employ subfigure
% captions (using the optional argument to \subfloat[]), but instead will
% reference/describe all of them (a), (b), etc., within the main caption.
% Be aware that for subfig.sty to generate the (a), (b), etc., subfigure
% labels, the optional argument to \subfloat must be present. If a
% subcaption is not desired, just leave its contents blank,
% e.g., \subfloat[].


% An example of a floating table. Note that, for IEEE style tables, the
% \caption command should come BEFORE the table and, given that table
% captions serve much like titles, are usually capitalized except for words
% such as a, an, and, as, at, but, by, for, in, nor, of, on, or, the, to
% and up, which are usually not capitalized unless they are the first or
% last word of the caption. Table text will default to \footnotesize as
% the IEEE normally uses this smaller font for tables.
% The \label must come after \caption as always.
%
%\begin{table}[!t]
%% increase table row spacing, adjust to taste
%\renewcommand{\arraystretch}{1.3}
% if using array.sty, it might be a good idea to tweak the value of
% \extrarowheight as needed to properly center the text within the cells
%\caption{An Example of a Table}
%\label{table_example}
%\centering
%% Some packages, such as MDW tools, offer better commands for making tables
%% than the plain LaTeX2e tabular which is used here.
%\begin{tabular}{|c||c|}
%\hline
%One & Two\\
%\hline
%Three & Four\\
%\hline
%\end{tabular}
%\end{table}


% Note that the IEEE does not put floats in the very first column
% - or typically anywhere on the first page for that matter. Also,
% in-text middle ("here") positioning is typically not used, but it
% is allowed and encouraged for Computer Society conferences (but
% not Computer Society journals). Most IEEE journals/conferences use
% top floats exclusively. 
% Note that, LaTeX2e, unlike IEEE journals/conferences, places
% footnotes above bottom floats. This can be corrected via the
% \fnbelowfloat command of the stfloats package.




\section{Conclusion}
The conclusion goes here.




% conference papers do not normally have an appendix


% use section* for acknowledgment
\section*{Acknowledgment}


The authors would like to thank...





% trigger a \newpage just before the given reference
% number - used to balance the columns on the last page
% adjust value as needed - may need to be readjusted if
% the document is modified later
%\IEEEtriggeratref{8}
% The "triggered" command can be changed if desired:
%\IEEEtriggercmd{\enlargethispage{-5in}}

% references section

% can use a bibliography generated by BibTeX as a .bbl file
% BibTeX documentation can be easily obtained at:
% http://mirror.ctan.org/biblio/bibtex/contrib/doc/
% The IEEEtran BibTeX style support page is at:
% http://www.michaelshell.org/tex/ieeetran/bibtex/
%\bibliographystyle{IEEEtran}
% argument is your BibTeX string definitions and bibliography database(s)
%\bibliography{IEEEabrv,../bib/paper}
%
% <OR> manually copy in the resultant .bbl file
% set second argument of \begin to the number of references
% (used to reserve space for the reference number labels box)
\begin{thebibliography}{1}
\bibitem{jiawei}
Jiawei Han, Micheline Kamber, Jian Pei. \emph{Data Mining Concepts and Techniques}, 3rd Edition, 2012.
\bibitem{Usama}
Usama Fayyad, Gregory Piatetsky-Shapiro, and Padhraic Smyth. \emph{From Data Mining to Knowledge Discovery in Databases}. AI Magazine Volume 17 Number 3, 1996.
\bibitem{fengchen}
Feng Chen, Pan Deng, Jianfu Wan and etc. \emph{Data Mning for the Internet of Things: Literature Review and cChallenges}. International Journal of Distributed Sensor Networks, 2015.
\bibitem{Bruno}
Bruno Pradel, Savaneary Sean,  Julien Delporte. \emph{A Case Study in a Recommender System Based on Purchase Data}. Proceedings of the 17th ACM SIGKDD international conference on Knowledge discovery and data mining, Pages 377-385, 2011.
\bibitem{Daniar}
Daniar Asanov. \emph{Algorithms and Methods in Recommender Systems.} Berlin Institute of Technology, Berlin, Germany, 2011.
\bibitem{imdb}
http://www.imdb.com/
\bibitem{movielens}
http://movielens.umn.edu/ 
\bibitem{IEEEhowto:kopka}
H.~Kopka and P.~W. Daly, \emph{A Guide to \LaTeX}, 3rd~ed.\hskip 1em plus
0.5em minus 0.4em\relax Harlow, England: Addison-Wesley, 1999.
\end{thebibliography}




% that's all folks
\end{document}


